%%
%%  Beinagrind fyrir Morpho handbók í LaTeX.
%%  Til að keyra þetta gegnum LaTeX forritið
%%  má t.d. nota pdflatex með eftirfarandi 
%%  skipun:
%%
%%       pdflatex handbok
%%
\documentclass[12pt,a4paper]{article}
\usepackage[icelandic]{babel}
\usepackage[pdftex]{hyperref}
%\usepackage{makeidx,smplindx,fancyhdr,graphicx,times,multicol,comment}
\usepackage{times}
\usepackage[T1]{fontenc}
\usepackage[rounded]{syntax}
\newenvironment{repnull}[0]{%
	\begin{stack}
	\\
	\begin{rep}
}{%
	\end{rep}
	\end{stack}
}
\newenvironment{málrit}[1]{%
	\par\noindent\begin{minipage}{\linewidth}\vspace{0.5em}\begin{quote}\noindent%
	\hspace*{-2em}\synt{#1}:\hfill\par%
	\noindent%
	\begin{minipage}{\linewidth}\begin{syntdiag}%
}{%
	\end{syntdiag}\end{minipage}\end{quote}\end{minipage}%
}

\begin{document}
\sloppy
\title{Handbók fyrir X}
\author{N. N.}
\maketitle

\begin{abstract}
Útdráttur...
\end{abstract}

\tableofcontents

\section{Inngangur}
\section{Notkun og uppsetning}
\section{Málfræði}
\subsection{Frumeiningar málsins}
\begin{málrit}{DIGIT}
	\begin{stack}
		`0'
	\\
		`1'
	\\
		`2'
	\\
		`3'
	\\
		`4'
	\\
		`5'
	\\
		`6'
	\\
		`7'
	\\
		`8'
	\\
		`9'
	\end{stack}
\end{málrit}

\begin{málrit}{FLOAT}
	\begin{rep}
		<DIGIT>	
	\end{rep}
	`.'
	\begin{rep}
		<DIGIT>	
	\end{rep}
	\begin{stack}
		
	\\
		\begin{stack}
			`e'
		\\
			`E'
		\end{stack}
		\begin{stack}
			`+'
		\\
			`-'
		\end{stack}
		\begin{rep}
			<DIGIT>	
		\end{rep}
	\end{stack}
\end{málrit}

\begin{málrit}{INT}
	\begin{rep}		
		<DIGIT>
	\end{rep}
\end{málrit}

\begin{málrit}{CHAR}
	`\''
	\begin{stack}
		\begin{stack}
			"any character except" `\''
		\\
			`\textbackslash '
		\end{stack}
	\\
		`\textbackslash ' `\''
	\\
		`\textbackslash ' `\textbackslash '
	\\
		`\textbackslash ' `n'
	\\
		`\textbackslash ' `r'
	\\
		`\textbackslash ' `t'
	\\
		`\textbackslash ' `b'
	\\
		`\textbackslash ' `f'
	\\
		`\textbackslash ' <octal>
	\\
		`\textbackslash ' <octal> <octal>
	\\
		`\textbackslash ' 
		\begin{stack}
			`0'
		\\
			`1'
		\\
			`2'
		\\
			`3'
		\end{stack}
		<octal> <octal>
	\end{stack}
	`\''
\end{málrit}

\begin{málrit}{STRING}
	`"'
	\begin{stack}
		
	\\
		\begin{rep}
			\begin{stack}		
				\begin{stack}
					"any character except" 
					`"'
				\\
					`\textbackslash '
				\end{stack}
			\\
				`\textbackslash ' `"'
			\\
				`\textbackslash ' `\textbackslash '
			\\
				`\textbackslash ' `n'
			\\
				`\textbackslash ' `r'
			\\
				`\textbackslash ' `t'
			\\
				`\textbackslash ' `b'
			\\
				`\textbackslash ' `f'
			\\
				`\textbackslash ' <octal>
			\\
				`\textbackslash ' <octal> <octal>
			\\
				`\textbackslash ' 
				\begin{stack}
					`0'
				\\
					`1'
				\\
					`2'
				\\
					`3'
				\end{stack}
				<octal> <octal>
			\end{stack}
		\end{rep}
	\end{stack}
	`"'
\end{málrit}

\begin{málrit}{octal}
	\begin{stack}
		`0'
	\\
		`1'
	\\
		`2'
	\\
		`3'
	\\
		`4'
	\\
		`5'
	\\
		`6'
	\\
		`7'
	\end{stack}
\end{málrit}

\begin{málrit}{literali}
	\begin{stack}
		<INT>
	\\
		<FLOAT>
	\\
		<CHAR>
	\\
		<STRING>
	\\
		`true'
	\\
		`false'
	\\
		`null'
	\end{stack}
\end{málrit}

\begin{málrit}{NAME}
	\begin{stack}
		<alpha>
	\\
		`_'
	\end{stack}	
	\begin{repnull}	
		\begin{stack}
			<alpha>
		\\		
			`_'
		\\
			<DIGIT>
		\end{stack}
	\end{repnull}
\end{málrit}

\begin{málrit}{alpha}
	"any Unicode character defined as Alphabetic"
\end{málrit}

\begin{málrit}{OP}
	\begin{rep}	
		\begin{stack}
			`?'
		\\
			`~'
		\\
			`^'
		\\
			`:'
		\\
			`|'
		\\
			`&'
		\\
			`<'
		\\
			`>'
		\\
			`='
		\\
			`!'
		\\
			`+'
		\\
			`-'
		\\
			`*'
		\\
			`/'
		\\
			`\%'
		\end{stack}
	\end{rep}
\end{málrit}

\subsubsection{Athugasemdir}
\subsubsection{Lykilorð}
\subsection{Mállýsing}
Hér verður sínd myndrit af helstu hlutum í málli okkar
% Orðum þetta bettur seina.
\subsubsection{Forrit}

\begin{málrit}{program}
	\begin{rep}
		<function>
	\end{rep}
\end{málrit}

\subsubsection{Föll}

\begin{málrit}{function}
	`fun' 
	\begin{stack}
		<NAME> \\ <OP>
	\end{stack}
	 `(' 
	\begin{repnull}
		<NAME> 
	\\
		`,'  
	\end{repnull}
	 `)' `=' <body>
\end{málrit}

\subsubsection{Stofnar}
\begin{málrit}{body}
	`{' 
	\begin{repnull}	
		<stmt>
	\\
		 `;'
	\end{repnull}
	`}' 
\end{málrit}

\begin{málrit}{stmt}
	\begin{stack}	
		`return' <expr>
	\\
		<expr>
	\\
		<decl>
	\\
		<whileexpr>
	\\
		<for_loop>
	\end{stack}
\end{málrit}

\subsubsection{Skilgreiningar}
\begin{málrit}{decls}
	`var' 
	\begin{rep}
		<NAME> 
		\begin{stack}
	
		\\		
			`=' <expr>
		\end{stack}
	\end{rep}
\end{málrit}


\subsubsection{Segðir}
\begin{málrit}{expr}
	\begin{stack}
		<LITERAL>
	\\
		<NAME>
		\begin{stack}
			
		\\
			`=' <expr>
		\\
			`(' 
			\begin{repnull}
				<expr> 
			\\
				`,'
			\end{repnull}
			`)'
		\end{stack}
	\\
		`(' <expr> `)'
	\\
		<ifexpr>
	\\	
		<condexpr>
	\\
		<matchexpr>
	\\
		<unop>
	\\
		<binop>
	\end{stack}
\end{málrit}

\begin{málrit}{ifexpr}
	`if' <expr> <body>
	\begin{stack}
		
	\\
		`else'
		\begin{stack}
			<body>
		\\
			<ifexpr>
		\end{stack}
	\end{stack}
\end{málrit}


\begin{málrit}{condexpr}
	`cond' `\{' 
	\begin{repnull}
		<expr> `=>' <body>
	\end{repnull}
	\begin{stack}
		
	\\
		<body>
	\end{stack}		 
	 `\}' 
\end{málrit}


\begin{málrit}{matchexpr}
	`match' expr `\{' 
	\begin{repnull}
		<expr> `=>' <body>  
	\end{repnull}	
	\begin{stack}
		
	\\
		<body>
	\end{stack}	
	`\}'
\end{málrit}

\begin{málrit}{unop}
	\begin{stack}
		<OP> <expr>
	\\
		`not' <expr>
	\end{stack}  
\end{málrit}

\begin{málrit}{binop}
	<expr> 
	\begin{stack}	
		`and'
	\\
		`or'
	\\
		<OP> 
	\end{stack}
	<expr>  
\end{málrit}

\begin{málrit}{whileexpr}
	`while' <expr> <body>  
\end{málrit}

\begin{málrit}{for_loop}
	`for' `(' 
	 \begin{stack}
	 	
	 \\
	 	<decal>
	 \end{stack}
	 `;' <expr> `;' 
	 \begin{stack}
	 	
	 \\
	 	<expr>
	 \end{stack}
	 `)' <body> 
\end{málrit}

\section{Merking málsins}
\subsection{Gildi}
\subsection{Breytur}
\subsection{Merking segða}
\subsubsection{null-segð}
\subsubsection{true-segð}
\subsubsection{false-segð}
\subsubsection{Heiltölusegð}
\subsubsection{Fleytitölusegð}
\subsubsection{Stafsegð}
\subsubsection{Strengsegð}
\subsubsection{Listasegð}
\subsubsection{return-segð}
\subsubsection{Röksegðir}
\subsubsection{Kallsegð}
\subsubsection{Tvíundaraðgerðir}
\subsubsection{Einundaraðgerðir}
\subsubsection{if-segð}
\subsubsection{while-segð}
xxx
\subsection{Föll og forrit}
xxx

\end{document}
